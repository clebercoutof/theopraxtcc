\chapter{Título do capítulo}\label{chapter:outro}

As organizações atuam em um ambiente global altamente competitivo, dinâmico, complexo e instável. Tais características denotam cenários de imprevisibilidade, onde todas as suas operações e atividades precisam ser vistas e revistas continuamente. Neste contexto do mundo dos negócios, é de fundamental importância que as organizações busquem estratégias eficazes para suas operações - planejamento, marketing, finanças, produção, qualidade e a logística - por serem cada vez mais importantes e por estarem relacionadas diretamente com as atividades fim dos variados sistemas produtivos.

Busca-se, portanto, reduzir custos em toda cadeia de valor e prover a satisfação dos clientes. Por esta ótica é possível entender o porquê da logística está em evidência na conjuntura atual de mercado. O fato desta ser amparada nos pilares: transportes, gestão de estoques, processamento de pedidos, atividades de apoio e ter como objetivo prover o cliente com os níveis de satisfação desejados pelos mesmos, faz com que a logística assuma um papel fundamental para as organizações: ser um componente capaz de gerar uma vantagem competitiva fundamental, ou seja, providenciar bens e serviços de maneira correta, em tempos e lugares exatos, na condição desejada e em menor custo possível.

Entretanto, quando se fala em melhoria da eficiência operacional na distribuição física, não é suficiente considerar apenas o meio de transporte mais utilizado no Brasil - o rodoviário; é preciso, analisar toda matriz de transporte disponível, para alcançar um serviço capaz de atender satisfatoriamente o canal de vendas. Essa visão considera cada etapa do processo de transporte, procurando sempre identificar as possíveis alternativas, muitas vezes descartadas ou mal exploradas.

Conforme Lambert, Stock e Vantine (1998), estima-se que no Brasil os gastos com atividades logísticas correspondam a 17\% do Produto Interno Bruto (PIB) e, na média, o transporte envolve 60\% dos custos logísticos das empresas. Estes dados justificam a necessidade de um sistema de transporte possuir mecanismos capazes de analisar quais opções de modais apresentam-se mais adequadas ao seu contexto de negócio. Ressalta-se ainda, que a seleção de modais afeta diretamente o preço dos produtos, as condições de entrega e a pontualidade, elementos estes considerados estratégicos para que o sistema alcance seu objetivo.

Segundo os mesmo autores, os cinco principais modais básicos são: rodoviário, ferroviário, aquaviário, dutoviário e aéreo, e sua importância relativa deve ser medida em termos de quilometragem do sistema, volume de tráfego, receita e natureza de composição. Justamente por tais questões que o modal aquaviário se configura como uma alternativa importante para a matriz de transportes brasileira, principalmente quando se fala em cabotagem, devido as características geográficas do Brasil.

Qualquer pesquisa efetuada no Brasil sobre modal aquaviário depara-se imediatamente com a seguinte questão fundamental: por que esse modal logístico é menos usado do que a lógica indicaria? O mesmo é verdade para o modal ferroviário. Por que são tão minguados face ao modal rodoviário? Qual a causa desse "transtorno dos modais", essa terrível modalidade de doença logística que aflige o Brasil e constitui uma de suas mais graves patologias?

Muitos estudos têm analisado essa síndrome de hipertrofia rodoviária e anemias ferroviária e aquaviária. Segundo a COPPEAD-UFRJ, o modal rodoviário responde por cerca de 60% de tudo que é transportado no país, enquanto em países de dimensões territoriais equivalentes ao Brasil, como os EUA e a Rússia, esses percentuais são, respectivamente, 35% e 19%.

Independentemente de comparações entre países, salta aos olhos imediatamente o absurdo de se transportar por caminhão cargas de São Paulo a Fortaleza ou Belém, em trajetos de mais de 3.000 km, quando existe a possibilidade de se usar a cabotagem, mais econômica e menos poluidora.

A considerável literatura técnica que versa sobre cabotagem refere-se quase exclusivamente à carga geral conteinerizada, porque é nessa área que reside o maior potencial de alteração de modal. Em matéria de granéis líquidos e sólidos, a maior parte do que pode ser transportado por cabotagem já o está sendo. Ademais, nesse caso, o transporte dutoviário é geralmente mais indicado.

As lamentações sobre a pouca utilização do modal de cabotagem devem ser entendidas como referentes sobretudo à submodalidade de carga geral. Mas há também absurdas quantidades de caminhões transportando granéis sólidos e líquidos a grandes distâncias nas estradas nacionais. A cabotagem, por sua vez, é parte de uma categoria mais ampla, o modal aquaviário ou hidroviário, dividido em dois submodais: cabotagem e transporte fluvial, esse chamado também de navegação interior, que envolve rios, canais e lagoas.

Portanto, pode-se afirmar que o processo de globalização, o transporte intermodal e as revoluções tecnológicas dos últimos anos na indústria de transporte fluvial resultaram em uma busca maior pela otimização das operações de empresas diretamente envolvidas nesse modal e, principalmente, na expansão das chamadas redes marítimas. No entanto, a falta de dados precisos sobre as relações entre os portos, vem impedindo uma maior aplicação da teoria de redes marítimas, que muitas vezes são analisadas por meio de estudos de caso de empresas ou regiões específicas.

A pesquisa e análise do referencial teórico especializado, na área de redes marítimas, demonstram que a grande maioria dos estudos e produção científica investigam pouco a resolução de problemas específicos sobre essas redes, especialmente observando importantes fatores diretamente relacionados a elas como movimentação de cargas, que interferem de maneira coordenada e agregadora no comportamento individual de cada nó dessa rede, ou seja, os portos.

Assim sendo, as análises de redes marítimas se constituem como um dos maiores campos a serem explorados pela evolução do conceito de redes sociais. Neste sentido, este projeto objetiva analisar a estrutura espacial e dinâmica regional da movimentação de contêineres por cabotagem entre os principais portos na costa brasileira, contribuindo com o estado da arte da seguinte maneira: (1) construindo e analisando redes marítimas, com a utilização de softwares especializados para modelagem computacional dessas; (2) hierarquizando os portos brasileiros através da metodologia AHP e (3) identificando possíveis formações de "clusters" entre os operadores logísticos envolvidos nessas redes através das chamadas constelações competitivas. O resultado da junção dessas análises poderá potencializar positivamente a cabotagem brasileira na obtenção de importantes impactos relacionados à sua gestão.
